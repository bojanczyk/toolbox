% !TEX root = ../main.tex


\exercise{zad:wqo-wqo}{
Show that the following conditions are equivalent for every quasi-order (a binary relation that is transitive and reflexive, but not necessarily anti-symmetric):
\begin{enumerate}
	\item  every infinite sequence contains an infinite subsequence that is  increasing (not necessarily strictly);
\item there are no infinite strictly decreasing sequences (i.e.~the quasi-order is well-founded) and no  infinite antichains (an antichain is  a set of pairwise incomparable elements);
\item every upward closed set is the upward closure of a finite set.
\end{enumerate}
A quasi-order that satisfies the above conditions is called a wqo.
}
{
	We prove the equivalences 1 $\Leftrightarrow$ 2 and 2 $\Leftrightarrow$ 3. The implications 1 $\Rightarrow$ 2 and 3 $\Rightarrow$ 2 are straightforward, so only prove the converses.
	\begin{itemize}
\item 2 $\Rightarrow$ 1. Take some infinite sequence of elements in the quasi-order. By the Ramsey Theorem, there is an infinite subsequence where either: (a) elements are strictly decreasing; (b) elements are pairwise incomparable; or (c) elements are (not necessarily strictly) increasing. Item 2 rules out cases (a) and (b), so we are left with case (c).
\item 2 $\Rightarrow$ 3. Take some upward closed set $U$, and consider the minimal elements. Because the quasi-order is well founded (by item 2), every element of $U$ is above some minimal element, and therefore $U$ is the upward closure of its minimal elements. If we take one minimal element for each equivalence class (i.e.~equivalence in the sense of both bigger and smaller), then we are left with an antichain, and this antichain is necessarily finite (by item 2).
	\end{itemize}
}


\exercise{zad:09-01}{
Which of the  following ordered sets are  wqo's?
\begin{enumerate}
  \item $\N^2$ with lexicographic order;
  \item $\{a, b\}^*$ with lexicographic order;
  \item $\N$ with divisibility order, i.e. $x$ smaller than $y$ if $x \ | \ y$;
  \item $\Sigma^*$ with prefix order;
  \item $\Sigma^*$ with infix order;
  \item line segments with an order: $[a,b]$ smaller than $[c,d]$ if $(b < c) \vee (a = c \wedge b \leq d)$;
  \item graphs with subgraph order (remove some edges and some vertices);
  \item trees with subtree order (remove some nodes, but keep the descendant ordering).
\end{enumerate}
}
{
Answers are the following.
\begin{enumerate}
  \item Yes. Any pair, which is dominating in Dickson's order is also dominating in lexicographic
  order. So by Dickson's lemma lexicographic order is also a wqo.
  \item No. An infinite descending sequence is of the form: $b, ab, aab, aaab, \ldots$.
  \item No. Prime numbers are an infinite antichain.
  \item No. An infinite descending sequence is of the form: $b, ab, aab, aaab, \ldots$.
  \item No. An infinite descending sequence is of the form: $bb, bab, baab, baaab, \ldots$.
  \item Yes. There is no infinite descending sequence, because sum $a+b$ is decreasing.
  There is also no infinite antichain. Assume there is one. Let $[a,b]$ be an element of it.
  Any $[c,d]$ in the antichain has to have $c \leq b$. So there are finitely many options for $c$,
  so some two segments in the antichain are of the form $[c,d_1]$ and $[c,d_2]$. However
  they have to be comparable, contradiction.
  \item No. Cycles $C_n$ for $n \geq 3$ are an infinite antichain. The same works also for induced subgraph order.
  \item No. An infinite antichain is formed by trees, which are paths of length $n$ such that both end vertices have
  additionally two neighbors (all together three neighbors). The same example works for induced subgraph
  order.
\end{enumerate}
}




\exercise{zad:09-08}{
Show that if $(X, \leq_X)$ and $(Y, \leq_Y)$ are both wqos then
also $(X \times Y, \leq)$ is wqo, where $(x, y) \leq (x', y') \iff x \leq_X x' \wedge y \leq_Y y'$.
}
{
Consider an infinite sequence of elements of $X \times Y$.
By the fact that $\leq_X$ is wqo there exists an infinite subsequence such that
first coordinates form an increasing subsequence. Then in that subsequence by the fact
that $\leq_Y$ is wqo there exists a dominating pair on second coordinates. This pair
is thus also a dominating pair in the order $\leq$.
}





\exercise{zad:09-02}{
Prove the Infinite Ramsey Theorem: in every infinite clique, with edges coloured on finitely many colours
there is an infinite monochromatic subgraph, i.e. subgraph such that all the edges in it are coloured by the same colour.
}
{
We sort vertices from left to right. First vertex has infinitely many outgoing edges, at least one color appears infinitely
many times. We choose such a color an leave only neighbors of this first vertex $v_1$ which have such colored edge
to $v_1$. Then we take $v_2$ (in the filtered sequence), there also exists a color such that $v_2$ has infinitely many
neighbors (to the right) with this color. We one more time filter vertices to the right of $v_2$
leaving only these which have appropriately colored edge with $v_2$.
In that way we also define $v_3, v_4, \ldots$. We always keep already defined vertices to the left untouched.
In that way we define $v_k$ for every $k \in \N$ so we have an infinite sequence of vertices $v_i$.
Every one has a distinguished color, so there exists a color in which there are infinitely many vertices.
They form a monochromatic clique.
}



\exercise{zad:09-03}{
Let $(X, \preceq)$ be a wqo. Show that there is no infinite growing sequence of upward-closed subsets $X$,
i.e. no sequence
\[
U_1 \subsetneq U_2 \subsetneq \ldots,
\]
s.t. for all $i \in \N$ set $U_i \subseteq X$ is upward-closed wrt. $\preceq$.
}
{
Assume towards contradiction that such an growing sequence exists.
Let $x_1 \in U_1$ and for $i > 1$ let $x_i \in U_i \setminus U_{i-1}$.
Because $\preceq$ is wqo there are some $i < j$ such that $x_i \preceq x_j$.
This means that $x_i \in U_i \subseteq U_{j-1}$ and by the fact that $U_{j-1}$ is upward-closed also $x_j \in U_{j-1}$.
Contradiction.
}


%mikolaj: w jakiej reprezentacji compute?
\exercise{zad:09-04}{
Show that given a  $d$-dimensional VAS and $s \in \N^d$, one can compute the  set of all configurations from which $s$ is coverable.
Hint: use Problem~\ref{zad:09-03}.
}
{

}






\exercise{zad:09-05}{
Show that given a vector addition system with a distinguished source configuration, one can decide if the set of configurations reachable from the source is finite.
}
{
We build a tree with root being vector $s$ and children of every vector $v$ being all
the $v+t$ for $t \in T$ such that $v+t \in \N^d$. However we can this tree in the following way.
If there is some vertex $v \in \N^d$ such that there exists its ancestor vertex $u \in \N^d$ with $u \preceq v$ then we do not
continue expanding vertex $v$. There are two cases. If $v$ is strictly bigger than $u$ on some coordinate by detecting dominating
pair $(u, v)$ on this path we know that reachability set is infinite. In the other case, if $u = v$, we know that it makes no sense
to expand this path, because we will not reach anything new. By Dickson's lemma we know that every path is finite.
Tree is finitely branching, therefore by K{\"{o}}nigs lemma the whole tree is finite. Therefore at some moment
we will compute the whole tree an algorithm will be finished. If all dominating pairs where $u = v$ then reachability set
is finite, otherwise it is infinite.
}






\exercise{zad:wqo-higman}{
Prove the following version of Higman's Lemma: if $\Sigma$ is a finite alphabet, then $\Sigma^*$ ordered by (not necessarily connected) subword is a wqo.}
{
}

\exercise{zad:wqo-lossy}{
Define a \emph{rewriting system} over an alphabet $\Sigma$ to be finite set of pairs $w \to v$ where $w,v \in \Sigma^*$. Define $\to^*$ to be the least binary relation on  $\Sigma^*$ which contains $\to$, is transitive,  and satisfies
\begin{align*}
  w \to^* v \qquad \text{implies} \qquad aw \to^* av \text{ and } wa \to^* va \qquad \text{for every }a \in \Sigma.
\end{align*}
There exist rewriting systems where $\to^*$ is an undecidable relation. Show that $\to^*$ is decidable if the rewriting system is \emph{lossy} in the following sense: for every letter $a \in \Sigma$, the rewriting system contains  $a \to \varepsilon$.}
{
}

\exercise{zad:wqo-zvass}{Define a \emph{$\Int$-vector addition system} in the same way as a vector addition system, except that configurations are vectors in $\Int^d$.  Show that the reachability problem is decidable, i.e.~one can decide if there is a run connecting two given configurations.}
{
}


\exercise{zad:vass}{Define a \emph{vector addition system with states} to be a finite set of states $Q$, a dimension $d$, and a finite set $\delta \subseteq Q \times \Int^d \times Q$. A configuration is an element of $Q \times \Nat^d$, and a transition is a pair 
\begin{align*}
(q,x) \to (p,y)	 \qquad \text{such that }(q,y-x,p) \in \delta.
\end{align*}
Show that the following problem is decidable: given states $p,q$ decide if there is a run from the configuration $(p,\bar 0)$  to some configuration with state $q$.
}
{
}


\exercise{zad:wqo-vass-exponential}{
\wojtek{Moze lepiej zrobic zadanie o zbiorze osiagalnosci przy ustalonym poczatku, a nie relacji osiagalnosci?
Latwiej sie mysli wg mnie.}
Find a vector addition system, say of dimension $d$, where the reachability relation
\begin{align*}
\set{(x,y) : \text{there is a run from from $x$ to $y$}} \subseteq \Nat^{2d}
\end{align*}
is not semilinear. Hint: use states and try to simulate exponentiation.
}
{
}

%mikolaj: czy dimension ma byc fixed czy nie?
\exercise{zad:wqo-vass-reach}{
Find a family of  vector addition systems with states, say of dimension $d$ (the dimension does not need to be fixed for the family), where the reachability set
\begin{align*}
\set{v : \text{there is a run from from the origin to $v$}} \subseteq \Nat^{d}
\end{align*}
is finite, but
\begin{enumerate}
  \item of doubly exponential size,
  \item of tower size
\end{enumerate}
with respect to the number of transitions.}
{
}
