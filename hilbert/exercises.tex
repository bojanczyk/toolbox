% !TEX root = ../main.tex

\exercise{zad:10-01}{
Which of the following structures are rings:
\begin{enumerate}
  \item $(\N, +, \cdot, 0)$;
  \item $(\Z, +, \cdot, 0)$;
  \item $(\R, +, \cdot, 0)$;
  \item $(\{0\}, +, +, 0)$;
  \item $(\Z[x_1, \ldots, x_n], +, \cdot, 0)$;
  \item $(\Q[x_1, \ldots, x_n], +, \cdot, 0)$;
  \item $(\Z, \max, +, 0)$;
  \item $(\N, \max, +, 0)$;  
  \item $(S_n, \cdot, \cdot, id)$;
  \item $(\Z_n, +, \cdot, 0)$ for $n \in \N$.
\end{enumerate}
}
{

}



\exercise{zad:10-02}{
Let $(R, +, \cdot, 0)$ be a ring. A subset $I \subseteq R$ is called an \emph{ideal}
if the following two conditions hold: 1) for all $i, j \in I$ it holds $i+j \in I$, 2) for all $i \in I, r \in R$ it holds $i \cdot r, r \cdot i \in I$.
Find all ideals in the following rings:
\begin{enumerate}
  \item $(\Z, +, \cdot, 0)$;
  \item $(\Q, +, \cdot, 0)$.
\end{enumerate}
Generalize the second case to any field.
}
{
Ideals in $\Z$: empty, singleton of zero and $n\Z = \{m \mid m \textup{divisible by} n\}$.
Ideals in any field: empty, singleton of zero and the whole field.
}




\exercise{zad:10-03}{
A ring congruence is an equivalence relation $\equiv$ such that
$x_1 \equiv y_1$, $x_2 \equiv y_2$ implies $x_1 * x_2 \equiv y_1 * y_2$ for $* \in \{+, \cdot\}$.
For an ideal $I \subseteq R$ and $r_1, r_2 \in R$ we say that $r_1 \equiv_I r_2$ if $r_1 - r_2 \in I$.
In case of the  ring of integers all the $\equiv_I$ are actually $\equiv_n$, so in particular ring congruences.
Show that for every ideal $I$, the  relation $\equiv_I$ is a ring congruence.
}
{
Let $x_1 \equiv_I y_1$, $x_2 \equiv_I y_2$.
Being a congruence wrt. $+$ is trivial by additivity of ideal elements.
Let us show that $\equiv_I$ is a congruence wrt. $\cdot$.
We have $x_1 x_2 - y_1 y_2 = (x_1 x_2 - x_1 y_2) + (x_1 y_2 - y_1 y_2) = x_1 (x_2 - y_2) + (x_1 - y_1) y_2$.
As $x_2 - y_2 \in I$ we have also that $x_1 (x_2 - y_2) \in I$.
As $x_1 - y_1 \in I$ we have also that $(x_1 - y_1) y_2 \in I$. Thus also $x_1 x_2 - y_1 y_2 \in I$, so indeed
$x_1 x_2 \equiv_I y_1 y_2$.
}


\exercise{zad:10-04}{
Show that every ideal in $\Q[x]$ is generated by one element.
}
{
First observe that ideals $\langle f, g \rangle$ and $\langle f, g - f \cdot r \rangle$ are equal for any polynomials $f, g \in \Q[x]$.
Then we can show that by adding new polynomial we still can keep the property that ideal is generated by one element.
If the added polynomial is not in the ideal generated by the previous polynomial then the degree of the new polynomial is smaller.
In that case we can show that the process terminates.
}


\exercise{zad:10-05}{
Is every ideal in the following rings generated by one element:
\begin{enumerate}
  \item $\Z[x]$?
  \item $\Q[x, y]$?
\end{enumerate}
}
{
In $\Z[x]$ no. Consider an ideal generated by elements $\{2, x\}$, it cannot be generated by one element.
In $\Q[x, y]$ also no. Ideal $\langle x, y \rangle$ cannot be generated by one element. This single generator cannot be a constant,
but also cannot contain $x$ (as it should generate $y$) and cannot contain $y$ (as it should generated $x$), which is impossible.
}




\exercise{zad:10-06}{
Is there a constant $c \in \N$ such that every ideal in $\Z[x]$ is generated by at most $c$ elements?
}
{
No. Consider an ideal generated by elements $\{n!, (n-1)! x, (n-2)!x^2, \ldots, 2 x^{n-1}, x^n\}$, it cannot be generated by less
than $n+1$ elements. TODO.
}




\exercise{zad:10-07}{
Show that for every ring $R$ the following conditions are equivalent:
\begin{enumerate}
  \item every ideal in $R$ is finitely generated;
  \item every growing sequence of ideals $I_1 \subsetneq I_2 \subsetneq \ldots$ is finite.
\end{enumerate}
}
{
First we show implication from 1. to 2. Assume towards contradiction that there is an ideal $I \subseteq R$,
which is not finitely generated and $r_1, r_2, \ldots, $ are some of its generators (maybe all).
Then we define $I_n = \langle r_1, \ldots, r_n \rangle$. It is easy to see that indeed $I_k \subsetneq I_{k+1}$
for every $k \in \N$, as in particular $r_{k+1} \in I_{k+1} \setminus I_k$.

For the second implication assume towards contradiction that there is an infinite growing sequence
of ideals in $R$, namely $I_1 \subsetneq I_2 \subsetneq \ldots$. Then consider $I = \bigcup_{n \in \N} I_n$, observe
that $I$ is also an ideal. Therefore $I$ is finitely generated, $I = \langle r_1, \ldots, r_k \rangle$.
There is thus such $n$ that all the $r_i$, for $i \in \{1, \ldots, k\}$ belong to $I_n$.
However than $I_n = I$ and thus $I_{n+1} = I_n$, contradiction.
}




\exercise{zad:10-08}{
Prove the Hilbert's Basis Theorem in the following formulation: if  $R$ is a ring where every ideal in $R$ is finitely generated,
then also every ideal in $R[x]$ is finitely generated.
}
{

}




\exercise{zad:10-09}{
Show that for every set $A \subseteq \definablereal$ the set $\mathsf{pol(A)}$ is an ideal.
}
{
It is enough to check the definition.
}




\exercise{zad:10-10}{
Consider the closure operation from $\P(\Q)$ to $\P(\Q)$ defined for $A \subseteq \Q$ as
$\bar A = \mathsf{zero}(\mathsf{pol(A)})$. Show that the following conditions are true for every $A, B \subseteq \Q$:
\begin{itemize}
  \item $A \subseteq \bar A$;
  \item if $A \subseteq B$ then $\bar A \subseteq \bar B$;
  \item $\bar A = \bar{ \bar A}$.
\end{itemize}
}
{
It boils down to checking definition.
}



\exercise{zad:hilbert-rational-one-basis}{
Consider the field of rational numbers $\Rat$. Show that for every finite set of variables $X$ and every ideal $I \subseteq \Rat[X]$,
there is an ideal $J \subseteq \Rat[x]$ generated by a single polynomial such that $\mathsf{zero}(I) = \mathsf{zero}(J)$.
}
{
The a vector $x \in \Rat^n$ is a simultaneous zero of two polynomials $p$ and $q$
if and only if it is a zero of the polynomial $p^2 + q^2$. This does not work for algebraic numbers.
}


\exercise{zad:hilbert-prod-inv}{
Assume that a closed set $A \subseteq \definablereal^n$ is represented by a finite basis for the ideal $\mathsf{pol}(A)$. Show that closed sets are effectively closed under products and inverse images of polynomials, i.e.~if $A,B$ are closed and $p$ is a polynomial, then the sets $A \times B$ and $p^{-1}(A)$ are closed, and their representations can be computed.
}
{ Suppose that $X,Y$ are finite disjoint sets of variables and let $A \subseteq \definablereal^X$ and $B \subseteq \definablereal^Y$ be closed sets that are zeros of  ideals generated by finite bases  $P \subseteq  \definablereal[X]$ and $Q \subseteq \definablereal[Y]$, respectively.  We can view $P \cup Q$ as a set of polynomials over variables $X \cup Y$ which do not use all variables. A simple check shows that $P \cup Q$ is a basis for $A \times B$.  For inverse images, a simple check shows that 
\begin{align*}
\set{p' \circ p : p '\in P} 	
\end{align*}
is a basis for the closed set $p^{-1}(A)$. 
}

\exercise{zad:hilbert-01}{
Show that the following problem is decidable: given a polynomial grammar and a finite set $X \subseteq \Rat$, decide if the  language generated by the grammar is equal to $X$.
}
{
}


