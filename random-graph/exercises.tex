\exercise{zad:finite-infinite-zero-one}{
  Show that the following conditions are equivalent for every first-order formula:
	\begin{enumerate}
		\item its limiting probability is one for finite graphs;
		\item it is true in the infinite random graph.
	\end{enumerate} 
}
{ The two conditions are also equivalent to:
\begin{itemize}
  \item[3.] for some $k$, it is true in all graphs with the $k$-extension property.
\end{itemize}
}


\exercise{zad:random-graph-decidable}{
  Show that the quantifier elimination in Theorem~\ref{thm:quantifier-elimination} is decidable, i.e.~the equivalent quantifier-free formula not only exists, but can be effectively computed.
}{}

\exercise{zad:random-graph-connected}{
  Prove that the infinite random graph is connected.
}{

}


\exercise{zad:scalar-product-graph}{
  For $n \in \set{1,2,\ldots}$ consider the following graph. The vertices are vectors $x \in \set{0,1}^n$, and two vertices are connected by an edge if their scalar product is odd (i.e.~nonzero if we work in the two-element field). Show that this graph has the $k$-extension property when $k = n^2$.
}
{

}

\exercise{zad:random-graph-mso}{
  Show that the zero-one law from Theorem~\ref{thm:zero-one-law} fails when, instead of first-order logic, we use monadic second-order logic. The latter is the extension of first-order logic where we also quantify over sets of vertices. (But not sets of pairs of vertices, or more complicated objects.)
}{
  s
}
