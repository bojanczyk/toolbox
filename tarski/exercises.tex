\exercise{zad:tarski-sinus}{ Show that adding the function $\sin(x)$ to the real numbers yields an undecidable theory.
}
{
Let $S$ be the set of numbers $x$ with $\sin(x)=0$. This set is exactly the integer multiples of $\pi$; and it is definable in first-order logic if $\sin$ is available.  The number $\pi$ is the least positive number in $S$, and is therefore also definable. Finally, the integers are exactly those numbers which are obtained by taking some $x \in S$ and dividing it by $\pi$. 
}


\exercise{zad:tarski-euclid}{ Show that the following structure has a decidable first-order theory: the universe consists of subsets of the Euclidean plane $\Real^2$ that are points, lines or circles, and there is a binary predicate for inclusion.
}
{

}


\exercise{zad:tarski-o-minimal}{ Show that every $X \subseteq \Real$ definable by a first-order formula with one free variable is a finite union of points and open intervals.
}
{
}




