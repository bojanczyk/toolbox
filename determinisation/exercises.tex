% !TEX root = ../main.tex

\exercise{zad:01-01}{
Are the following languages $\omega$-regular (i.e.~recognised by nondeterministic B\"uchi automata)?
\begin{enumerate}
  \item $\omega$-words which have infinitely many prefixes in a  fixed regular language of finite words $L \subseteq \Sigma^*$;
  \item $\omega$-words with infinitely many infixes of the form $a b^p a$, where $p$ is prime;
  \item $\omega$-words with  infinitely many infixes of the form $a b^n a$, where $n$ is even.
\end{enumerate}
}
{
Solutions of the points:
\begin{enumerate}
  \item YES. Let $\A$ be an automaton for $L$. We make the same automaton with the same final states and B\"{u}chi
  acceptance condition.
  \item NO. Assume that yes and $\A$ is an automaton for this language. Let $\A$ has $n$ states
  and let $p > n$ be some prime number. The word $(b^p a)^\omega$ belongs to $L$,
  so $\A$ has an accepting run on it.
  Note that in every block $b^p$ some two states are the same. We pump the part $b^k$
  between them $p$ times, so that the block has now the length $b^{p + kp}$, which is not prime.
  The now word is accepted by $\A$, because it has an accepting run (which came out from the pumping of
  an old run), but no block has prime length.
  \item YES. It suffices to count length of the block $b^k$ modulo $2$ and go into an accepting state on $a$
  which finishes such a block.
\end{enumerate}
}

\exercise{zad:01-02}{
Call an $\omega$-word  \emph{ultimately periodic} if it is of the form $uv^\omega$ for some finite words $u,v$. Show that if an $\omega$-regular language  is nonempty, then it contains an ultimately periodic word.}
{
Towards contradiction assume that $L \neq \emptyset$. Then there is some infinite word $w \in L$.
Word $w$ visits infinitely many times accepting state (in automaton for $L$),
so in particular visits some concrete accepting state $q$ infinitely many times.
In particular $w = x y z$ such that $x$ goes from initial state to state $q$
and $y$ goes from state $q$ to itself. Then also word $x y^\omega$ belongs to $L$ and is ultimately periodic.
}

\exercise{zad:01-03}{
Let $UP$ be the set of ultimately periodic words. 
Let $K$ and $L$ be $\omega$-regular languages. Show that if $L \cap UP = K \cap UP$ then $K = L$.
}
{
Assume towards contradiction that $K \neq L$. Without loss of generality assume that $K \setminus L \neq \emptyset$
(symmetric case would be $L \setminus K \neq \emptyset$). Clearly $K \setminus L$ is regular.
Thus $(K \setminus L) \cap UP \neq \emptyset$, which implies that $K \cap UP \neq L \cap UP$. Contradiction.
}

\exercise{zad:01-04}{
Are the following languages $\omega$-regular?
\begin{enumerate}
  \item $\omega$-words with arbitrarily long infixes belonging to a fixed regular language of finite words $L$;
  \item $\omega$-words which have infinitely many  prefixes in a fixed language of finite words $L \subseteq \Sigma^*$ (not necessarily regular).
\end{enumerate}
}
{
Solutions of the points:
\begin{enumerate}
  \item NO. Consider $L = ab^*a$. This language contains no ultimately periodic word, so it would have to be
  empty to be regular.
  \item NO. Let us fix some infinite word $u$, which is not ultimately periodic. Let $L$ be all its
  prefixes. Prefixes of $v$ belong infinitely often to $L$ if and only if $v = u$. However, the language $\{u\}$ is not
  regular. By the way, note that the language $\{w\}$ is regular iff $w$ is ultimately periodic.
\end{enumerate}
}

\exercise{zad:01-05}{
Show that the language of words ``there exists a letter $b$'' cannot be accepted by a nondeterministic
automaton with the B\"{u}chi acceptance condition, where all the states are accepting (but possibly transitions
over some letters in some states are missing).
}
{
By reading $a^k$ for any $k \in \N$ we cannot get blocked. Therefore word $a^\omega$ also cannot
get blocked, which means that it is accepted by the considered automaton, contradiction.
}

\exercise{zad:01-06}{
Show that the language ``finitely many occurrences of letter $a$'' cannot be accepted by a deterministic
automaton with the B\"{u}chi acceptance condition.
}
{
Assume towards a contradiction that it is accepted by some automaton with $n$ states.
Let $w = (a b^n)^{\omega}$. For any prefix of it of the form $(a b^n)^k$ there should be an accepting
state among the last $n+1$ states. Indeed, otherwise a word $(a b^n)^k b^\omega$ would not be accepted.
Therefore a run over $w$ visits infinitely many times an accepting state, which means that $w$ is accepted.
On the other hand it does not belong to the language, as it has infinitely many letters $a$. Contradiction.
}

\exercise{zad:01-07}{
Show that every language accepted by a nondeterministic automaton with the Muller acceptance
condition is also accepted by some nondeterministic automaton with the B\"{u}chi acceptance condition.
}
{
We have an automaton $\A$ with Muller condition, we will be trying to make an automaton $\A'$ with
B\"{u}chi condition such that $L(\A') = L(\A)$.
For every $S \in \F$ we will do a separate gadget in automaton $\A'$ such that acceptance in this gadget
is if and only if $\inf(\rho) = S$. At the beginning we just make a copy of $\A$, but such that no states
are accepting. Beside that for every $S \in \F$ we add a gadget $\A_S$. The idea is that automaton $\A'$
will jump into the gadget $\A_S$ if it wants to choose that $\inf(\rho) = S$ and  this is the  moment
from which on only states from $S$ will occur. Let $S = \{q_1, \ldots, q_k\}$.

Observe now that if $\inf(\rho) = S$ and there are no states outside of $S$ in $\rho$ then states occurring in $\rho$
have also an infinite subsequence of the form $(q_1 q_2 \cdots q_k)^\omega$. Thus we can just investigate whether
there exists such a subsequence. Gadget $A_\S$ will be the following. It contains $|S|$ copies of $\A$:
$\A_{S, 1}, \ldots, \A_{S, |S|}$. The copy $\A_{S, i}$ has only one accepting state: $q_i$.
In the copy $\A_{S, i}$ transitions are like in $\A$ with the only exception that from state $q_i$ we go
to the next copy: $\A_{S, (i+1) \mod |S|}$.
Now we can easily observe that $\rho$ visits infinitely many times accepting state iff it infinitely many
times changes a copy. Therefore it has a subsequence of states of the form $(q_1 q_2 \cdots q_k)^\omega$,
so indeed $\inf(\rho) = S$.
}

%
%\exercise{zad:01-08}{
%\todo{ale w wykladzie uzywa sie tranzycji}
%Consider an acceptance condition where instead of states visited infinitely often, transitions visited
%infinitely often along the run are taken into account. How does it affect the expressive power of automata in the case
%of
%\begin{enumerate}
%  \item B\"{u}chi acceptance condition;
%  \item Muller acceptance condition.
%\end{enumerate}
%}
%{
%In both cases (B\"{u}chi and Muller) expressivity does not change.
%Therefore we have to prove four facts: (1) condition with states can be implemented on transitions,
%(2) condition with transitions can be implemented on states, both points for both B\"{u}chi and Muller
%acceptance conditions.
%
%Let us start
%\begin{enumerate}
%  \item We first implement states on transitions for B\"{u}chi condition.
%  It is very easy, simply these transitions are final which finish into previously final states.
%  \item Now we implement transitions on states for B\"{u}chi condition. Let language $L$ be accepted
%  by automaton $\A$ with B\"{u}chi acceptance condition on transitions. We make an automaton $\A'$,
%  which has two copies of every state in $\A$. To one copy go all the accepting transitions, while to another one
%  go all the non accepting ones. The outgoing transitions are identical in both copies, the same as in $\A$.
%  All the copies with accepting incoming transitions are final, while the other not (some of the final states may
%  not be reachable, but this is not the problem).
%  \item Now we implement states on transitions for Muller acceptance condition. This is also easy, set of transitions
%  is accepting iff the set of states into which they go is accepting.
%  \item Now we implement transitions on states for Muller acceptance conditions.
%  Let automaton $\A$ with Muller condition on transitions accept $L = L(\A)$. We make $\A'$ as follows.
%  Every state of $\A$ is split into as many copies in $\A'$ as it has incoming transitions in $\A$.
%  The state of $\A'$ is accepting if the incoming transition in $\A$ was accepting.
%\end{enumerate}
%}

\exercise{zad:01-09}{
Show that nonemptiness is decidable for automata with the Muller acceptance condition.
}
{
It is enough to check whether for some $S \in \F$ there exists a run $\rho$ of an automaton in which $\inf(\rho) = S$,
where $\inf(\rho)$ is the set of states which occur infinitely often in $\rho$.
We do this separately for every $S \in \F$. We check whether the graph with only states from $S$ is strongly connected
and whether some state from $S$ is reachable from some initial state (now in the situation where we have all the states).
}



\exercise{zad:01-10}{
Define a metric on $\omega$-words by
\begin{align*}
d(u,v) = \frac{1}{2^{\diff(u,v)}},
\end{align*}
where $\diff(u,v)$ is the smallest
position where  $u$ and $v$ have different labels. A language $L$ is called \emph{open} (in this metric) if for every $w \in L$
there exists some open ball centered in $w$ that is included in $L$ (standard definition).
Prove that the following conditions are equivalent for an $\omega$-regular language $L$:
\begin{enumerate}
  \item  is open;
  \item  is of the form $K \Sigma^\omega$ for some $K \subseteq \Sigma^*$;
  \item  is of the form $K \Sigma^\omega$ for some regular $K \subseteq \Sigma^*$.
\end{enumerate}
}
{
We will first show equivalence of (1) and (2), so two implications.

First $(1) \Rightarrow (2)$. If $L$ is open then around every $w \in L$ there is a ball $L_w$ with positive radius
such that $L_w \subseteq L$. Therefore $L = \bigcup_{w \in L} L_w$. Observe however that $L_w = f(w) \Sigma^\omega$,
where $f(w)$ is some finite prefix of $w$.Thus $L = \bigcup_{w \in L} f(w) \Sigma^\omega$, therefore $L = K \Sigma^\omega$
for $K = \bigcup_{w \in L} f(w)$.

Now $(2) \Rightarrow (1)$. Let $w \in L = K \Sigma^\omega$. Therefore $w = u v$, where $u \in K$.
Thus for any $v' \in \Sigma^\omega$ we have $u v' \in L$. Therefore the ball centered in $w$ and radius equal
$\frac{1}{2^{|u|+1}}$ is included in $L$.

Now we will show equivalence of (1) and (3). Implication $(3) \Rightarrow (1)$ works the same as
$(2) \Rightarrow (1)$, so we will focus on implication $(1) \Rightarrow (3)$.

Let us consider an automaton $\A_L$ of language $L$, where $\A_L$ is a deterministic
automaton with Muller acceptance condition.
Let $Q_U \subseteq Q$ be a subset of states such that $q \in Q_U$ iff $L(q) = \Sigma^\omega$.
Let $L' \subseteq \Sigma^*$ be the set of finite words accepted by automaton $\A_L$,
where final states are $Q_U$. We will show that $L = L' \Sigma^\omega$.
We have $L \supseteq L' \Sigma^\omega$, because it is easy to find a run in $\A_L$ for every word from $L' \Sigma^\omega$.
We simply first go like in automaton for $L'$ and then arbitrarily and this is an accepting run for $\A_L$.
On the other hand let us take $w \in L$ and its run. After some its prefix we will already be in the ball $L_w$,
which means that independent of the rest of the word, everything is accepted, so we are in the state from $Q_U$.
This gives us a division into prefix from $L'$ and the rest.
}



\exercise{zad:01-12}{
Which of the following candidates for a  Myhill-Nerode congruence indeed have the property:
$\sim_L$ has finite index if and only if $L$ is $\omega$-regular
\begin{enumerate}
  \item an equivalence relation $\sim_L$ on $\Sigma^*$ where  $u \sim_L v$ is defined by
  \begin{align*}
uw \in L \iff vw \in L \qquad \text{for all $w \in \Sigma^\omega$}
\end{align*}
 \item an equivalence relation $\sim_L$ on $\Sigma^\omega$ where  $u \sim_L v$ is defined by
  \begin{align*}
wu \in L \iff wv \in L \qquad \text{for all $w \in \Sigma^*$}
\end{align*}
  \item an equivalence relation $\sim_L$ on $\Sigma^*$ where  $u \sim_L v$ is defined by
  \begin{align*}
 \text{and} \begin{cases}
uw \in L \iff vw \in L  & \text{for all $w \in \Sigma^\omega$}\\
s (u t)^\omega \in L \iff s (v t)^\omega \in L &\text{for all $s,t \in \Sigma^*$}
\end{cases}
\end{align*}
\end{enumerate}
}
{None. In all cases if $L$ is regular then $\sim_L$ has a finite index.
We will first show this. Consider an automaton $\A$ with B\"{u}chi condition for $L$.
In $\sim_L$ we just have to remember
\begin{enumerate}
  \item which states of $\A$ one can reach via a word $u$;
  \item from which states of $\A$ there exists an accepting run via a word $u$;
  \item as in 1. and in 2. and additionally for which pair of states $p, q \in Q(\A)$ there exists a run via a word $u$
  which goes from $p$ to $q$ and a) has an accepting state b) has no accepting state.
\end{enumerate}
Now we show that there exist nonregular languages $L$ such that $\sim_L$ has finite index.
\begin{enumerate}
  \item every prefix independent language, for example: language from exercise 1.2
  ($u$ contains infinitely many infixes of the form
  $ab^pa$, where $p$ is prime). For such a language $\sim_L$ has only one equivalence class;
  \item also every prefix independent language, as language from 1.2 works. For such a language
  relation $\sim_L$ has two classes ($wu \in L$ or $wu \not\in L$, this does not depend on $w$);
  \item language similar like from exercise 1.4 works ($u$ contains arbitrary long infixes of the form $ba^*b$).
  It is not regular. Definitely $uw \in L$ does not depend on $u$. It is enough to know whether $u$ contains any $b$
  (if yes then $s (v t)^\omega \not \in L$) and additionally if not whether $u$ is empty (it matters for empty $t$).
\end{enumerate}
In general there exists no reasonable relation with this property.
}



%%% Local Variables:
%%% TeX-master: "../EN_main"
%%% End:
